\recipient{Prof. Fernando Boada}{Department of Radiology \\ Stanford Medicine \bigskip} % Letter rec
\date{\today} % Letter date
\opening{Dear Prof. Boada,} % Opening greeting
\closing{Sincerely yours,} % Closing phrase
\enclosure[Attached]{curriculum vit\ae{} and list of publications} % List of enclosed documents

\makelettertitle % Print letter title

\begin{minipage}{\textwidth} %create minipage
\setlength{\parskip}{0.2cm}% restore the value 

I would like to apply for the postdoctoral position focussing on
anatomically-guided image reconstruction posted in the ISMRM career center
on 28 Jan 2022.

Currently, I am a PostDoc in the medical image reconstruction group of 
Prof. Johan Nuyts at KU Leuven in Belgium mainly working on joint image 
reconstruction in hybrid PET/MR imaging.
One my main focusses of the last year was improving and evaluating 
iterative PET reconstruction using anatomical priors derived from 
high-resolution structural MR scans.
In our 2018 IEEE TMI publication ``Evaluation of Parallel Level Sets and Bowsher’s 
Method as Segmentation-Free Anatomical Priors for Time-of-Flight PET Reconstruction'',
we could e.g. show that using structural priors improves the bias-noise
trade-off in brain PET reconstructions.
Very recently, we could also show that similar image quality can be achieved
post reconstruction using a deep convolutional neural network
(see e.g. ``Approximating anatomically-guided PET reconstruction in image space 
using a convolutional neural network'', NeuroImage 2021).
Together with colleagues from Siemens Healthcare, we were able to translate
this approach into a work-in-progress package that is now available on
all Siemens mMR PET/MR scanners.

Since sodium brain MR images suffer from silimar problems as PET images - namely
relatively low spatial resolution and high noise - we also investigated the
concept of anatomy-guided reconstruction using a similar prior for this
reconstruction problem. 
First results of this approach were recently presented at the 2020 IEEE MIC
and the 2021 ISMRM conferences.

Based on your publication record, it is clear that you are a very well-known
researcher in the field of MR imaging and I would like to contribute with
my experience in iterative image reconstruction using structural
(anatomical) priors and my strong interest in high performance computing in the
context of inverse problems in medical imaging to your future research. 

In the meantime, thank you in advance for considering my application. 

\end{minipage}

\makeletterclosing % Print letter signature

\textbf{List of referees}
\bigskip
\begin{itemize}
\item Prof. Johan Nuyts - supervisor during my PostDoc time at KU Leuven, Belgium - \href{mailto:johan.nuyts@uzleuven.be}{johan.nuyts@uzleuven.be}
\item Prof. J\"org van den Hoff - supervisor during my PhD time at HZDR / TU Dresden, Germany - \href{mailto:j.van_den_hoff@hzdr.de}{j.van\_den\_hoff@hzdr.de}
\item Prof. Martin Holler - collaborator at university of Graz, Austria - \href{mailto:martin.holler@uni-graz.at}{martin.holler@uni-graz.at}
\end{itemize}

\clearpage
