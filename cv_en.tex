%% start of file `template.tex'.
%% Copyright 2006-2012 Xavier Danaux (xdanaux@gmail.com).
%
% This work may be distributed and/or modified under the
% conditions of the LaTeX Project Public License version 1.3c,
% available at http://www.latex-project.org/lppl/.


\documentclass[10pt,a4paper,sans]{moderncv}   % possible options include font size ('10pt', '11pt' and '12pt'), 
                                              % paper size ('a4paper', 'letterpaper', 'a5paper', 'legalpaper', 
                                              % 'executivepaper' and 'landscape') and font family ('sans' and 'roman')

% moderncv themes
\moderncvstyle{classic}                        % style options are 'casual' (default) and 'classic' 
\moderncvcolor{blue}                          % color options 'blue' (default), 'orange', 'green', 'red', 'purple', 'grey' and 'black'

% character encoding
\usepackage[utf8]{inputenc}
\DeclareUnicodeCharacter{2032}{$^\prime$}

% adjust the page margins
\usepackage[scale=0.8]{geometry}
\usepackage[english]{babel}
%\usepackage{mathptmx}
\usepackage[sfdefault,light]{roboto}
\usepackage{pgf}

\setlength{\hintscolumnwidth}{3.5cm}          % Breite links
\setlength{\separatorcolumnwidth}{0.75cm}     % Abstand 
\setlength{\maincolumnwidth}{7.0cm}             % Breite rechts

% personal data
\firstname{Georg}
\familyname{\textsc{Schramm}}
\address{190 E Okeefe Street}{94205 Menlo Park, CA, US} % optional, remove the line if not wanted
\phone{+1~650~643~6259}                         % optional, remove the line if not wanted
\email{georg.schramm@posteo.net}                 % optional, remove the line if not wanted
\homepage{gschramm.github.io}                      % optional, remove the line if not wanted


\renewcommand\sectionfont{\Large\sffamily}

%----------------------------------------------------------------------------------
\begin{document}

\small

%\recipient{Prof. Ir. Johan Nuyts}{Nuclear Medicine and Medical Imaging Research Center \\ K.U.Leuven \bigskip} % Letter recipient
\date{\today} % Letter date
\opening{Dear Prof. Nuyts,} % Opening greeting
\closing{Sincerely yours,} % Closing phrase
\enclosure[Attached]{curriculum vit\ae{} and list of publications} % List of enclosed documents

\makelettertitle % Print letter title

\begin{minipage}{\textwidth} %create minipage
\setlength{\parskip}{0.2cm}% restore the value 

I am applying for the postdoctoral position of the NIH project
"Advancing MR and PET Through Synergistic Simultaneous Acquisition and 
Joint Reconstruction" in your group.

Currently and until March 2015 I am a PhD student in the group of Prof. 
J\"org van den Hoff at Helmholtz-Zentrum Dresden-Rossendorf (HZDR) dealing with 
the evaluation and improvement of MR-based attenuation correction (MRAC) for 
PET reconstruction.
Your postdoc position strongly fits to me because during my time as a PhD 
student at HZDR, I was able to acquire detailed knowledge about hybrid PET/MR 
imaging using one of the first PET/MR scanners world-wide.

Within the three years of the PhD thesis, I have developed strong skills
in PET/MR imaging.
These skills include:

\begin{itemize}
  \item daily clinical PET/MRI image acquisition, phantom 
        measurements, scanner calibration and quality assurance
  \item medical image analysis and comparison as demonstrated in my
        Magn Reson Mater publication in 2012
  \item development of new algorithms for image segmentation as demonstrated
        in the my publications about truncation and metal artifact in MRAC
  \item experience with simulations about simultaneous reconstruction of
        activity and attenuation as shown in the outlook of my PhD thesis
  \item strong skills in computation in various languages (python, R, C++)
\end{itemize}

I know that your group is one of the most experienced groups in the field of
PET image reconstruction, as you have shown e.g. in the development of MLAA
or MLACF.
Thus, working in your group would be a great chance for me to expand my 
knowledge in PET reconstruction.

In the meantime, I would like to thank you for your time and effort of reviewing 
my application. 

\end{minipage}

\makeletterclosing % Print letter signature

\clearpage


\maketitle

\section{Personal Details}
\cvitem{date of birth}{08 April 1987}
\cvitem{place of birth}{G\"orlitz, Germany}
\cvitem{nationality}{German}


\section{Education}
\cventry{Jan 2015}{PhD in medical imaging}
        {}{}{TU Dresden, Germany}{Thesis: "Evaluation and Improvement of MR-based attenuation 
                                  correction in PET/MRI." \\ final mark: summa cum laude,
                                  \href{https://d-nb.info/1067633286}{\color{color1}{link 
                                  (German National Library)}}}
        %{}{}{ \\
        %     topic: }
\cventry{Apr 2011}{Master in (nuclear and particle) physics}
        {}{}{TU Dresden, Germany}
        {Thesis: "Simulation and analysis of neutron capture and photon scattering experiments."}
%\cventry{Sep 1997 - Jun 2005}{Abitur (equivalent to A-levels)}{Augustum Annengymnasium G\"orlitz, Germany}
%{}{}{final mark: 1.1}


\section{Experiences}
\cventry{since Aug 2022}{Visiting Instructor}{}{Stanford University School of Medicine, Department of Radiology}{}{}
\cventry{Apr 2015 - Jul 2022}{Postdoctoral researcher}{}{KU Leuven, Belgium, Department of Imaging and 
                                                    Pathology, Division of Nuclear Medicine}
        {}
        {As a PostDoc in the lab of Prof. Johan Nuyts, I am investigating
        joint advanced method for iterative PET image reconstruction and the application
        of deep learning in PET reconstruction and image analysis.
        Moreover, I am heavily interested and involved in the translation of our research
        into clinical routine.}
\cventry{Jan 2015 - Mar 2015}{Scientist}{}{Helmholtz-Zentrum Dresden-Rossendorf (HZDR), Institute
                                  for Radiopharmaceutical Cancer Research}
        {}{}
\cventry{May 2011 - Jan 2015}{PhD student}{}{HZDR, Institute
                                  for Radiopharmaceutical Cancer Research}
        {}
        {As a PhD student in the lab of Prof. J\"org van den Hoff, I was evaluating
         and improving whole-body MR-based attenuation correction using one of
         the first combined PET/MR systems world-wide.}
\cventry{2021}{Member of the local organizing comittee for the 16th Virtual International Meeting on
               Fully 3D Image Reconstruction in Radiology and Nuclear Medicine}{}{}{}
        {As a member of the organizing comittee, I was responsible for organizing the virtual
         poster session on gather.town as well as editing the conference proceedings submitted
         to \href{https://arxiv.org/abs/2110.04143}{\color{color1}{arvix}}}
\cventry{since Apr 2019}{Active member in the KU Leuven PostDoc Society}{}{}{}
        {I am involved in organizing career and networking events for PostDocs and
         in the preparation of a PostDoc charta for KU Leuven.}
\cventry{Apr 2013 - Feb 2015}{member of the management board of Werkstatt BigBand Dresden e.V.}{}{}{}
       {In our student big band, I was organizing concerts, rehearsal weekends and finances.}
\cventry{Sep 2009 - Mar 2010}{Semester abroad}{}{University of Sheffield, UK}
        {}
        {During my Erasmus semester in Sheffield, I was studying astronomy and applied mathematics}
\cventry{Aug 2008 - Jul 2009}{Student research assistant}{}{HZDR Institute of Radiation Physics}
        {}{As an assistant, I was analysing neutron TOF and tranmission data and typesetting
        a lecture manuscript in latex.}

%----------------------------------------------------------------------------------
\section{Languages}
\cvitem{German}{native}
\cvitem{English}{fluent}
\cvitem{Dutch}{basic}

%----------------------------------------------------------------------------------
\newpage

\section{Teaching}
\cvitem{since 2017}{Techniques and technologies in Nuclear Medicine (assistant for Prof. J.~Nuyts)}
\cvitem{since 2017}{Medical Imaging (assistant for Prof. P.~Suetens and Prof. F.~Maes)}


%----------------------------------------------------------------------------------

\section{Awards}
\cventry{Nov 2019}{Best Poster Award 2nd place}
{}{}{}{Synergistic Reconstruction Symposium, Chester}
\cventry{Mar 2015}{PhD Award}
{}{}{}{Yearly award for the best PhD thesis at HZDR}
\cventry{Mar 2014}{Award for notable achievements in nuclear medicine imaging}
{}{}{}{German Society of Nuclear Medicine}
\cventry{Mar 2014}{Travel grant for RSNA 2014 for the best oral presentation of 
a young investigator.}
{}{}{}{Annual meeting of the German Society of Nuclear Medicine}
\cventry{May 2012}{Award for the best oral presentation of 
a young investigator.}{}{}{}{International conference on PET/MRI and SPECT/MRI. La Biodola, Italy}
\cventry{Jan 2012}{Ehrenfried Walter von Tschirnhaus Urkunde}{}
{}{}{Yearly given to the five best graduates of the faculty of science at TU Dresden}

%----------------------------------------------------------------------------------

\section{Invited Talks}
\cventry{Nov 2019}{State of the art of AI for medical image
reconstruction and corrections}{}{}{}
{IEEE MIC 2019 workshop: Emergence and perspectives of artificial intelligence (AI) methods in radiation-based imaging sciences, Manchester}
\cventry{Oct 2017}{MR-based attenuation correction for the body}{}{}{}
{Annual congress of the European Association for Nuclear Medicine, Vienna}
\cventry{Sep 2017}{Positron Emission Tomography - an introduction and overview about current developments}{}{}{}
{International workshop on positron studies on defects 2017, Dresden}

%----------------------------------------------------------------------------------

\section{Research Interests}
\cvitem{PET}{PET image reconstruction}
\cvitem{}{quantitative PET imaging}
\cvitem{}{PET image analysis}
\cvitem{}{hybrid PET/MR imaging}
\cvitem{}{Deep learning in medical image reconstruction and analysis}
\cvitem{Reviewer for}{J Nucl Med, Eur J Nucl Med, IEEE TMI, Eur J Nucl Med Phys, Physica Medica}
\cvitem{Associate Editor for}{Eur J Nucl Med Phys}

%----------------------------------------------------------------------------------

\section{Skills}
\cvitem{programming}{Python, keras, tensorflow, pytorch, IDL, matlab, R, C, C++, bash, git, cmake, openmp, cuda, the dicom standard, \href{https://github.com/gschramm}{\color{color1}{github link}}}
\cvitem{mathematics}{numerics, inverse problems and convex optimization in medical imaging}
\cvitem{clinical PET imaging}{more than 8 years of experience in clinical operation of a PET/MRI scanner}

%----------------------------------------------------------------------------------


\bigskip
\bigskip

\begin{flushright}
Leuven, \today
\end{flushright}

\newpage

%publication list
\footnotesize
\section{Publication records} 
Google scholar profile \href{https://scholar.google.de/citations?hl=en&user=_txZ90cAAAAJ&view_op=list_works}{\color{color1}{link}} \\

ORCID ID \href{https://orcid.org/0000-0002-2251-3195}{\color{color1}{link}}

\input{publications_gs.tex}

\bigskip

\end{document}
